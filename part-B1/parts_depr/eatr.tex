\subsection{Bibliometrics and summary of publication activities}
I have authored over 40 journal publications and over 90 conference publications in the fields of machine learning (reinforcement learning, variational inference, meta-learning, time-series modelling, learning physics simulations), robotics (robot reinforcement learning, imitation learning, movement representations, dynamics learning) and computer vision (meta-learning and 6D pose estimation). Many of these papers have been published at Tier-1 conferences (NeurIPS, ICLR, ICML, RSS, CoRL, ICRA, IROS) or journals (JMLR, MLJ, IJRR, RAL, Auro). 

\subsection{10 research highlights}

I now present my achievements through a selection of my publications, in
reverse chronological order. My name appears in bold, names of PhD
students I supervised in italics. In all papers, I am the last author and the main supervisor.

\begin{asparaitem}
\AtNextCite{\defcounter{maxnames}{99}}
\item \fullcite{shaj2023mts3}. 
We developed a new multi-time scale state space model that defines the new state of the art for long-term prediction performance on many time-series prediction tasks. This approach provides a basis for the long-term prediction models used for model-based fine-tuning within SMARTeach (see WP3.c).

\AtNextCite{\defcounter{maxnames}{99}}
\item \fullcite{blessing2023information}. We propose a new curriculum learning method for learning mixture of expert models, which can represent versatile behavior in an efficient manner. The method will provide the basis for the hierarchical manipulation skill learning methods developed in SMARTeach-

\item \fullcite{li2023prodmp}. In this paper, we introduce a unification of ProMPs  and DMPs which inherits the advantages of both methods, i.e., smooth motion generation due to the dynamical system view and an easy way to obtain probabilistic motion representations. This work will serve as basic motion representation used in SMARTeach.  

\item \fullcite{otto2023deep}


\item \fullcite{Gao_2022_CVPR}

\item \fullcite{otto2021differentiable}

\item \fullcite{volpp2021bayesian}

\item \fullcite{becker19RKN}

\item \fullcite{Huettenrauch19}

\item \fullcite{paraschos2013probabilistic}





\end{asparaitem}

\subsection{Invited presentations}

\noindent\hskip -.5cm\begin{tabular}{>{\bfseries}rll}
  \toprule
  \multicolumn{1}{c}{\bfseries Date}&{\bfseries Location}&{\bfseries
Context or inviting institution}\\
  \midrule
  ~\\
  \bottomrule
\end{tabular}

\subsection{Awards}

\cvsection{Major collaborations}
\cvitem{\textbf{Tele-op + AR Interfaces}}{ Collaborations with Dr. Ayse Kucukyilmaz (Lecturer at the University of Nottingham), Ass. Prof. Rudolf Lioutikov (KIT), and Prof. Sören Hohmann (KIT).}
\cvitem{\textbf{Computer Vision:}} {Collaborations with Prof. Markus Vincze (TU Wien), and Prof. Jürgen Beyerer (KIT + Fraunhofer IOSB).}
\cvitem{\textbf{Simulation:}} {Collaborations with Prof. Luise Kärger (KIT) on using ML for enhancing simulations.}
\cvitem{\textbf{Reinforcement Learning:}} {
Collaborations with Prof. Jan Peters (TU Darmstadt), Ass. Prof. Jeannette Bohg (Stanford), Ass. Prof. Oliver Kroemer (CMU), and Ass. Prof. Joni Pajarinnen (Aalto University).}
\cvitem{\textbf{Robotics Use Cases:}}{ My collaborations extend to specialists in various robotics applications, including Prof. Franziska Mathis-Ullrich (University of Erlangen) for surgical robotics, Prof. Tamim Asfour (KIT) on service robotics within the Jubot Project, and Prof. Gisela Lanza (KIT) and Prof. Jürgen Fleischer in the newly funded DFG collaborative research center (CRC) "Circular Factory," focusing on robot learning techniques for the circular economy sector. I also maintain strong connections with the UK nuclear robotics sector through figures like Prof. Rustam Stolkin (University of Birmingham) and Dr. Simon Watson (University of Manchester), a collaborator within the RAICAM MSCA doctoral training center.}
\cvitem{\textbf{Industry}}{
In the industrial sector, I have established relationships with Bosch Research and several local startups, including Artiminds (robotic integrator) and Telekinesis (computer vision, teleoperation).}
